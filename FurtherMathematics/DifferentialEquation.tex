\section{微分方程}

\begin{spacing}{2.2}
    {\color{red} $ y' = f(x) \cdot g(y)$}型:$\displaystyle \Rightarrow \frac{\mathrm{d}y}{g(y)} = f(x)\mathrm{d}x \Rightarrow \int{\frac{\mathrm{d}y}{g(y)}} = \int{f(x)}\mathrm{d}x $

    {\color{red} $y' = f(ax + by + c)$}型$\colon$令$\displaystyle u = ax +by + c \Rightarrow u' = a + bf'(u) \Rightarrow \frac{\mathrm{d}x}{a + bf(u)} = \mathrm{d}x \Rightarrow \int{\frac{\mathrm{d}x}{a + bf(u)}} = \int\mathrm{d}x$

    {\color{red} $y' = f(\frac{y}{x})$}型$\colon$令$\frac{y}{x} = u \Rightarrow y = ux \Rightarrow \frac{\mathrm{d}y}{\mathrm{d}x} = u + x\frac{\mathrm{d}u}{\mathrm{d}x}$原方程$\displaystyle x\frac{\mathrm{d}u}{\mathrm{d}x} + u = f(u) \Rightarrow \frac{\mathrm{d}u}{f(u)-u} = \frac{\mathrm{d}x}{x} \Rightarrow \int{\frac{\mathrm{d}u}{f(u)-u}} = \int{\frac{\mathrm{d}x}{x}}$

    {\color{red} $\frac{1}{y'} = f(\frac{x}{y})$}型$\colon$令$\frac{x}{y} = u \Rightarrow x = uy \Rightarrow \frac{\mathrm{d}x}{\mathrm{d}y} = u + y\frac{\mathrm{d}u}{\mathrm{d}y}$原方程$\displaystyle y\frac{\mathrm{d}u}{\mathrm{d}y} + u = f(u) \Rightarrow \frac{\mathrm{d}u}{f(u)-u} = \frac{\mathrm{d}y}{y} \Rightarrow \int{\frac{\mathrm{d}u}{f(u)-u}} = \int{\frac{\mathrm{d}y}{y}}$

    {\color{red} $y' + p(x)y = q(x)$}型$\colon$方程两边同时乘上$\displaystyle e^{\int{p(x)\mathrm{d}x}} \Rightarrow e^{\int{p(x)\mathrm{d}x}} \cdot y' + e^{\int{p(x)\mathrm{d}x}}p(x) \cdot y = e^{\int{p(x)\mathrm{d}x}} \cdot q(x) \Rightarrow \left[e^{\int{p(x)\mathrm{d}x}} \cdot y\right]' = e^{\int{p(x)\mathrm{d}x}} \cdot q(x) \Rightarrow e^{\int{p(x)\mathrm{d}x}} \cdot y = \int{e^{\int{p(x)\mathrm{d}x}} \cdot q(x)}\mathrm{d}x + C$

    \noindent 得$\displaystyle y = e^{-\int{p(x)\mathrm{d}x}}\left[\int{e^{\int{p(x)\mathrm{d}x}} \cdot q(x)\mathrm{d}x} + C\right]$

    {\color{red} $y' + p(x)y = q(x)y^n$}型$\colon$先变形到$y^{-n} \cdot y' + p(x)y^{1-n} = q(x) \stackrel{z = y^{1-n}}{\Longrightarrow}$得$\frac{\mathrm{d}z}{\mathrm{d}x} = (1 - n)y^{-n}\frac{\mathrm{d}y}{\mathrm{d}x}$,则$\displaystyle \int \frac{1}{1 - n}\frac{\mathrm{d}z}{\mathrm{d}x} + p(x)z = q(x)$

    {\color{red} $y''=f(x, y')$}型$\colon$令$y' = p \Rightarrow y'' = p' \Rightarrow \frac{\mathrm{d}p}{\mathrm{d}x} = f(x, p)$
    若解得$p = \varphi(x, C_1)$即$y' = \varphi(x, C_1)$则通解为$\displaystyle y = \int{\varphi(x, C_1)\mathrm{d}x} +C_2$

    {\color{red} $y''=f(y', y'')$}型$\colon$令$y' = p \Rightarrow y'' = \frac{\mathrm{}{d}p}{\mathrm{d}x} = \frac{\mathrm{d}p}{\mathrm{d}y} \cdot \frac{\mathrm{d}y}{\mathrm{d}x} = \frac{\mathrm{d}p}{\mathrm{d}y}p$得$p\frac{\mathrm{d}p}{\mathrm{d}y} = f(y, p)$若解得$p = \varphi(y, C_1)$则由$p = \frac{\mathrm{d}y}{\mathrm{d}x} \Rightarrow \frac{\mathrm{d}y}{\mathrm{d}x} = \varphi(y, C_1)$分离变量得$\displaystyle \frac{\mathrm{d}y}{\varphi(y, C_1)} = \mathrm{d}x \Rightarrow \int{\frac{\mathrm{d}y}{\varphi(y, C_1)}} = x + C_2$

    {\color{red} $y'' + py' + qy = f(x)$}型$\colon$

    \noindent $\left\{
        \begin{array}{l}
            \lambda^2 + p\lambda + q = 0 \Rightarrow \lambda_1, \lambda_2 \Rightarrow \mbox{写出齐次方程的通解}\\
            \mbox{设特解}y'' \Rightarrow \mbox{回代,求待定系数} \Rightarrow \mbox{特解} \\
        \end{array}
    \right.
    \Rightarrow \mbox{写出通解}$

    {\color{red} $y'' + py' + qy = f_1(x) + f_2(x)$}型$\colon$

    \noindent $\left\{
        \begin{array}{l}
            \mbox{写}\lambda^2 + px + q = 0 \Rightarrow \mbox{齐次方程的通解} \\
            \begin{array}{l}
                y'' + py' + q = f_1(x) \mbox{写特解}y_{1}^{*} \\
                y'' + py' + qy = f_2(x) \mbox{写特解}y_{2}^{*} \\
            \end{array} \Rightarrow \mbox{故}y_{1}^{*} + y_{2}^{*}\mbox{为特解}
        \end{array}
    \right. \Rightarrow \mbox{通解}$

    \noindent $\mbox{(1)齐次方程的通解}$

    \noindent $\begin{array}{ll}
        p^2-4q > 0,\mbox{即} \lambda_1 \neq \lambda_2 & \mbox{通解为}y = C_1e^{\lambda_1x} + C_2e^{\lambda_2x}\\
        p^2-4q = 0,\mbox{即} \lambda_1 = \lambda_2 = \lambda & \mbox{通解为}y = (C_1 + C_2x)e^{\lambda x}\\
        p^2-4q < 0,\mbox{共轭复根为}\alpha \pm \beta \mathrm{i} & \mbox{通解为}y = e^{\alpha x}(C_1\cos{\beta x} + C_2\sin{\beta x})\\
    \end{array}$

    \noindent $\mbox{(2)非齐次方程的特解}$

    \noindent $\ding{1} \mbox{自由项}f(x) = P_n(x)e^{ax}\mbox{时,特解}y^{*} = e^{ax}Q_n(x)x^k$

    \noindent $\left\{\begin{array}{l}
        e^{ax}\mbox{照抄}\\
        Q_n(x)\mbox{为}x\mbox{的}n\mbox{次一般多项式}\\
        k=\left\{
        \begin{array}{ll}
            0 & \alpha \neq \lambda_1,\alpha \neq \lambda_2\\
            1 & \alpha \neq \lambda_1\mbox{或}\alpha \neq \lambda_2\\
            2 & \alpha = \lambda_1 = \lambda_2\\
        \end{array}
        \right.\\
    \end{array}\right.$
\end{spacing}
